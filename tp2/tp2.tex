\documentclass[a4paper,11pt]{article}
 \usepackage[utf8]{inputenc}
 \usepackage[T1]{fontenc}
 \usepackage[normalem]{ulem}
 \usepackage[french]{babel}
 \usepackage{verbatim}
 \usepackage{graphicx}

\title{Compte-rendu du TP 2 d'Informatique Embarqué}
\author{Daniel RESENDE - M2 MIC}
\date{Vendredi 13 octobre 2017}
\begin{document}

\maketitle

\section{Gestion du temps}
Temps estimé : 5 heures

Temps réalisé : 4 heures.

\section{Implémentation du programme}

%\subsection*{Choix de l'implémentation}

%\subsection*{Problèmes rencontrés lors de l'implémentation}

\subsection*{Choix du prototype de la fonction \texttt{colimacon}}
	J'avais déjà choisi ce prototype de fonction lors du TP1. Ce prototype de fonction permet de remplir un tableau de taille n x m et de renvoyer un code d'erreur.

Parallélisation du programme :
L'implémentation ne le permet pas.

\section{Performances du programme}

\paragraph{Nombre d'accès mémoire au tableau :}
Étant donnée que l'on ne fait qu'écrire un entier dans chaque case, il y a donc n x m accès mémoire.


\end{document}